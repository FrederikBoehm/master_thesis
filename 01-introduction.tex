\chapter{Introduction}
\label{chap:intro}

\section{Motivation}
\label{sect:motivation}
In recent years computer hardware became increasingly powerful, allowing now even real time raytracing with the introduction of the RTX technology by NVIDIA in 2018 \cite[pp. 31 - 38]{turing_whitepaper}.
In parallel to the advances in computer hardware the scenes to be rendered also became more and more complex reaching more than 70 GB for a single shot in the film Coco \cite{pixarxpu}.
While high detailed geometry and textures are required for close-up shots they impair performance when rendering them at large distances like in a landscape scene.
This is because rays have to be intersected with all triangles in an AABB regardless of the distance to the camera.
Additionally having high detailed geometry and textures at distance is prone to aliasing due to undersampling \cite[pp. 409-410]{pbr}.
According to the Nyquist-Shannon sampling theorem we can avoid this aliasing as long as our sampling frequency is twice as high as the maximum frequency in the signal \cite[p. 11]{shannonsampling}.
Therefore we could increase the number of rays that are traced per pixel.
However this would lead to even worse performance.
The other option is to reduce the frequency of the signal.
For geometry this is typically accomplished by simplifying a detailed mesh in a preprocess to rendering \cite[pp. 706 - 712]{realtime} or by generating new triangles on the fly using tessellation \cite[pp. 767 - 781]{realtime}.
The usual approach to reduce the frequency in textures is to use mipmapping \cite[pp. 265 - 267]{fundamentals}.
In this thesis we explore another approach for level of detail rendering by representing geometry and color by volumes.

\section{Challenges}


\section{Contributions}
The thesis provides answers to the following questions:
\begin{itemize}
	\item Can meshes be represented by volumes without a noticeable loss in quality?
	\item How does the performance change when representing meshes by volumes at different distances?
\end{itemize}
