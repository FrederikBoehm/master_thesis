\chapter*{Abstract}
Level of detail (LOD) rendering is a widely used technique in modern renderers to reduce render times and aliasing in scenes with far viewing distances.
Various approaches exist that apply this technique, but they all have in common to render less detailed models at increasing distances.
This thesis proposes an approach for \acs{lod} rendering by representing surface meshes with volumes.
We use ray casting to filter the surface meshes and obtain volume representations with differing detail.
The optimal \ac{lod} is selected using a heuristic which compares the number of voxels the camera sees with the number of pixels the volume covers on the image plane.
On a large forest scene, we evaluate different ratios of these numbers and different minimum voxel sizes in order to find a tradeoff between a good render performance and image quality.
We also show the implications on the memory consumption of these different configurations.
Additionally, tests regarding quality and performance are performed on single model instances.
All rendering is done in a custom developed physically based path tracer.