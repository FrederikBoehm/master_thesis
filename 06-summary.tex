\chapter{Summary}
\label{chap:summary}
This thesis presented an approach for filtering textured surface meshes, thereby converting them into volumes.
The resulting volumetric representations occupy 22\% less disk space.
We further developed a scene generator that uniformly distributes models on a circular area.
It selects an appropriate \ac{lod} based on the number of pixels that a volume covers compared to the number of voxels visible from the camera.
We then rendered a forest scene containing only meshes and scenes with different \ac{lod} selection strategies in our custom physically based path tracer.
The results showed that we can decrease render times by up to 17\% while introducing a \FLIP error of 0.038.
Since we did not aim for a lossless representation, this is an acceptable error.
We can achieve lower \FLIP errors at the cost of higher render times.
Using our \acsp{lod} also reduces the memory consumption by $23-98\%$, depending on the configuration for \ac{lod} selection.
Additionally, we performed tests on single model instances and quantified the distance at which there is a performance benefit of the volumetric representations.
Within the scope of the single instance tests, we also showed the error introduced by coarser \ac{lod} at a fixed render distance.
We can therefore conclude that volumetric representations of surface meshes can reduce render times while introducing noticeable differences in the image quality in a side by side comparison.